\documentclass{article}
\usepackage{enumitem}
\usepackage{amsmath}
\usepackage{amssymb}
\usepackage{listings}
\lstset{basicstyle=\ttfamily}
\usepackage[margin=0.75in]{geometry}
\usepackage[english]{babel}
\usepackage{float}
\usepackage{xcolor}
\def\partialv{\partial}
\def\deltav{\delta}
\def\){\Big)}
\def\({\Big(}

\title{Test}
\author{Julia Bruce-Robertson}
\date{\today}

\begin{document}
\maketitle


CURRENT ISSUES:

•	Replacement issues\\
•	ISSUES WITH NEGATIVES!!!!\\

QUESTION — should outside brackets have to be \begin{verbatim}\(\end{verbatim}????\\

\begin{equation}
\partial_{\mu} A \(Y \partial_{\nu}h^{\mu \nu} + X \partial^{\mu} h^{\nu }_{\nu}\)
\end{equation}

\begin{equation}
\partial_{\mu}\(YA \partial_{\nu}h^{\mu \nu} +XA \partial^{\mu}h_{\nu}^{\nu} \)
\end{equation}


Contract:
\begin{equation}
\partial_{\mu} A \(Y \partial_{\nu}h^{\mu \nu} + X \partial^{\mu} h^{\nu }_{\nu}\)
\end{equation}




\section{Representation}
Reading in and understanding equations (display initial equation)\\


Reading in signs between brackets
\begin{equation}
\(a\) + \(b\) 
\rightarrow
\(a \)+\(b \)
\end{equation}

\begin{equation}
\(a \)-\(b \) 
\rightarrow 
\(a \)-\(b \)
\end{equation}

basic reading in signs inside brackets
\begin{equation}
b_{\gamma} - g_{j} 
\rightarrow 
\( b_{\gamma} - g_{j} \)
\end{equation}

inside and outside brackets
\begin{equation}
A^{\mu} + \(A^{\gamma} - B^{\zeta}\) 
\rightarrow
A^{\mu} +\( A^{\gamma} - B^{\zeta} \)
\end{equation}


\begin{equation}
A^{\mu} - \(A^{\gamma} - B^{\zeta}\) 
\rightarrow
A^{\mu} - \( A^{\gamma} - B^{\zeta} \)
\end{equation}


with partial derivatives
\begin{equation}
\partial_{\alpha} \( a +\( b_{\gamma} - g\)\) 
\rightarrow 
\partial_{\alpha} \(a +\( b_{\gamma} -g \)\)
\end{equation}

multiplied by a tensor

\begin{equation}
A^{\mu} + C^{\epsilon}\(A^{\gamma} - B^{\zeta}\) 
\rightarrow
A^{\mu} + C^{\epsilon} \( A^{\gamma} - B^{\zeta} \)
\end{equation}


complicated bracket nesting
\begin{equation}
\(\(a \)- \(b - \(c \)\)\)+\(\(a - d\) - b\) \rightarrow
\(\(a \)- \(b - \(c \)\)\)+\(\(a -d \)-b \)
\end{equation}

complicated bracket nesting and tensor indices
\begin{equation}
\(\(a \)- \(b - \(c \)\)\)+\(\(a^{\gamma} - d^{\gamma}\) - b\) 
\rightarrow
 \(\(a \)- \(b - \(c \)\)\)+\(\( a^{\gamma} - d^{\gamma} \)-b \)
\end{equation}

complicated bracket nesting and tensor indices, multiplication, and partials
\begin{multline}
\partial_{\zeta}\(G^{\gamma} \) \partial^{\zeta} \square \(\(A_{} \)- \(B^{\kappa}_{\kappa} - \(C^{} \)\)\)+\(\(A^{\gamma} - D^{\gamma}\) - B^{\alpha \gamma}_{\alpha}\)  \\
\rightarrow
\partial_{\zeta}\( G^{\gamma} \) \partial^{\zeta} \square \(\( A^{} \)- \( B_{\kappa}^{\kappa} - \( C^{} \)\)\)+\(\( A^{\gamma} - D^{\gamma} \)- B_{\alpha}^{\alpha \gamma} \)
\end{multline}


with equals sign
\begin{equation}
G^{\mu} = A^{\mu} + C^{\mu \epsilon}\(A_{\epsilon} - B_{\epsilon}\) 
\rightarrow
G^{\mu} = A^{\mu} + C^{\mu \epsilon} \( A_{\epsilon} - B_{\epsilon} \)
\end{equation}


with begin equation command
\begin{verbatim}
\begin{equation}
G^{\mu} = A^{\mu} + C^{\mu \epsilon}\(A_{\epsilon} - B_{\epsilon}\) 
\end{equation}
->
\begin{equation}
G^{\mu} = A^{\mu} + C^{\mu \epsilon} \( A_{\epsilon} - B_{\epsilon} \)
\end{equation}
\end{verbatim}


with begin multline command (but short so it changes to equation)
\begin{verbatim}
\begin{multline}
G^{\mu} = A^{\mu} + C^{\mu \epsilon}\(A_{\epsilon} - B_{\epsilon}\) 
\end{multline}
->
\begin{equation}
G^{\mu} = A^{\mu} + C^{\mu \epsilon} \( A_{\epsilon} - B_{\epsilon} \)
\end{equation}
\end{verbatim}

with begin multline command and long enough to stay multline
\begin{verbatim}
\begin{multline}
G^{\mu} = A^{\mu} + C^{\mu \epsilon}\(A_{\epsilon} - B_{\epsilon}\) + A^{\mu} 
+ C^{\mu \epsilon}\(A_{\epsilon} - B_{\epsilon}\)  + A^{\mu} 
+ C^{\mu \epsilon}\(A_{\epsilon} - B_{\epsilon}\) + A^{\mu} 
+ C^{\mu \epsilon}\(A_{\epsilon} - B_{\epsilon}\)  + A^{\mu} 
+\\ C^{\mu \epsilon}\(A_{\epsilon} - B_{\epsilon}\) + A^{\mu} 
+ C^{\mu \epsilon}\(A_{\epsilon} - B_{\epsilon}\) + A^{\mu} 
+ C^{\mu \epsilon}\(A_{\epsilon} - B_{\epsilon}\) + A^{\mu} 
+ C^{\mu \epsilon}\(A_{\epsilon} - B_{\epsilon}\) + A^{\mu} 
+\\ C^{\mu \epsilon}\(A_{\epsilon} - B_{\epsilon}\) + A^{\mu} 
+ C^{\mu \epsilon}\(A_{\epsilon} - B_{\epsilon}\) + A^{\mu} 
+ C^{\mu \epsilon}\(A_{\epsilon} - B_{\epsilon}\) 
\end{multline}
->
\begin{multline}
G^{\mu} = A^{\mu} + C^{\mu \epsilon} \( A_{\epsilon} - B_{\epsilon} \)+ A^{\mu} 
+ C^{\mu \epsilon} \( A_{\epsilon} - B_{\epsilon} \)+ A^{\mu} 
+ C^{\mu \epsilon} \( A_{\epsilon} - B_{\epsilon} \)+ A^{\mu} \\
+ C^{\mu \epsilon} \( A_{\epsilon} - B_{\epsilon} \)+ A^{\mu} 
+ C^{\mu \epsilon} \( A_{\epsilon} - B_{\epsilon} \)+ A^{\mu} 
+ C^{\mu \epsilon} \( A_{\epsilon} - B_{\epsilon} \)+ A^{\mu} \\
+ C^{\mu \epsilon} \( A_{\epsilon} - B_{\epsilon} \)+ A^{\mu} 
+ C^{\mu \epsilon} \( A_{\epsilon} - B_{\epsilon} \)+ A^{\mu} 
+ C^{\mu \epsilon} \( A_{\epsilon} - B_{\epsilon} \)+ A^{\mu} \\
+ C^{\mu \epsilon} \( A_{\epsilon} - B_{\epsilon} \)+ A^{\mu} 
+ C^{\mu \epsilon} \( A_{\epsilon} - B_{\epsilon} \)
\end{multline}
\end{verbatim}

also print above to test what it looks like visually: output is

\begin{multline}
G^{\mu} = A^{\mu} + C^{\mu \epsilon} \( A_{\epsilon} - B_{\epsilon} \)+ A^{\mu} + C^{\mu \epsilon} \( A_{\epsilon} - B_{\epsilon} \)+ A^{\mu} + C^{\mu \epsilon} \( A_{\epsilon} - B_{\epsilon} \)+ A^{\mu} \\
+ C^{\mu \epsilon} \( A_{\epsilon} - B_{\epsilon} \)+ A^{\mu} + C^{\mu \epsilon} \( A_{\epsilon} - B_{\epsilon} \)+ A^{\mu} + C^{\mu \epsilon} \( A_{\epsilon} - B_{\epsilon} \)+ A^{\mu} \\
+ C^{\mu \epsilon} \( A_{\epsilon} - B_{\epsilon} \)+ A^{\mu} + C^{\mu \epsilon} \( A_{\epsilon} - B_{\epsilon} \)+ A^{\mu} + C^{\mu \epsilon} \( A_{\epsilon} - B_{\epsilon} \)+ A^{\mu} \\
+ C^{\mu \epsilon} \( A_{\epsilon} - B_{\epsilon} \)+ A^{\mu} + C^{\mu \epsilon} \( A_{\epsilon} - B_{\epsilon} \)
\end{multline}

each line could be longer before wrapping but I'm pretty happy with how it looks for now\\

with etas and deltas and fractions

\begin{equation}
7X\eta_{\mu \nu} \delta_{\nu}^{\gamma}\square A^{\mu} + \frac{1}{3}\delta_{\mu}^{\gamma}C^{\mu \epsilon}\(A_{\epsilon} - B_{\epsilon}\) 
\rightarrow
7 X \delta_{\nu}^{\gamma} \eta_{\mu \nu} \square A^{\mu} +\frac{1}{3} \delta_{\mu}^{\gamma} C^{\mu \epsilon} \( A_{\epsilon} - B_{\epsilon} \)
\end{equation}


multiple coefficients 
\begin{equation}
\(XY Z \partial_{\nu}h^{\mu \nu} + 6bX \partial^{\mu} h^{\nu }_{\nu}\) 
\rightarrow
\(XY Z \partial_{\nu}h^{\mu \nu} +6 bX \partial^{\mu}h_{\nu}^{\nu} \)
\end{equation}

multiple coefficients in unusual order
\begin{equation}
\(XY \partial_{\nu}h^{\mu \nu}Z + b6X \partial^{\mu} h^{\nu }_{\nu}\) 
\rightarrow
\(XY Z \partial_{\nu}h^{\mu \nu} +6 bX \partial^{\mu}h_{\nu}^{\nu} \)
\end{equation}



{\color{red}
breaks (is this good or no??)
\begin{equation}
A^{\alpha}\frac{1}{3}B_{\gamma} \rightarrow \textrm{
Equation: pop from empty list}
\end{equation}

this one seems to work…
\begin{equation}
A^{\alpha}3B_{\gamma} \rightarrow 
 3 A^{\alpha} B_{\gamma}
 \end{equation}
 }

QUESTION: does multiplication always need brackets? If no, when does it need them? Why does the whole number work but the fraction breaks….\\

Please email jbrucero@uwo.ca for if you think this is a bug\\

*****things that shouldn’t work\\
*** uneven brackets\\
\begin{equation}
\(A + B \rightarrow WORKING (though the message could be greatly improved)
"Equation:
Please email jbrucero@uwo.ca for if you think this is a bug”
\end{equation}

*** uneven index brackets
\begin{equation}
A^\{\gamma  \rightarrow Working in that it should error but again the issue could probably be caught earlier and a better message given)
"Equation: string index out of range
Please email jbrucero@uwo.ca for if you think this is a bug”
\end{equation}

\begin{equation}
A^{\gamma} + B^\{ \rightarrow this one reads the equation fine - is this an issue???
\( A^{\gamma} + B^{} \)
\end{equation}


%%%%%%%%%%%%%%%%%%%%%%%%%%%%%%%%%%%%%%%%%%%%

\section{Multiply}

%%%
 \subsection{FOIL out terms, distributing derivatives when necessary (recommended)}
Basic case:
\begin{equation}
\(A^{}\) \partial_{\mu} \(Y \partial_{\nu}h^{\mu \nu} + X \partial^{\mu} h^{\nu }_{\nu}\)  	
\rightarrow
\(Y A^{} \partial_{\nu}\partial_{\mu}h^{\mu \nu} +X A^{} \partial^{\mu}\partial_{\mu}h_{\nu}^{\nu} \)
\end{equation}


Also need to test adding summations, with and without partials, and subtracting summations, with and without partials
%%%
\subsection{ distribute partial derivatives}
Basic case:
\begin{equation}
\partial_{\mu} \(Y \partial_{\nu}h^{\mu \nu} + X \partial^{\mu} h^{\nu }_{\nu}\)  \rightarrow \(Y \partial_{\nu}\partial_{\mu}h^{\mu \nu} +X \partial^{\mu}\partial_{\mu}h_{\nu}^{\nu} \)
\end{equation}

** number coefficients
\begin{equation}
\partial_{\mu} \(5 \partial_{\nu}h^{\mu \nu} + 3 \partial^{\mu} h^{\nu }_{\nu}\)  \rightarrow  \(5 \partial_{\nu}\partial_{\mu}h^{\mu \nu} +3 \partial^{\mu}\partial_{\mu}h_{\nu}^{\nu} \)
\end{equation}

** mixed coefficients
\begin{equation}
\partial_{\mu} \(5 \partial_{\nu}h^{\mu \nu} + V \partial^{\mu} h^{\nu }_{\nu}\) \rightarrow \(5 \partial_{\nu}\partial_{\mu}h^{\mu \nu} +V \partial^{\mu}\partial_{\mu}h_{\nu}^{\nu} \)
\end{equation}

** product rule
\begin{equation}
\partial_{\mu} \( A^{\gamma}B^{\zeta}\) \rightarrow
 \( \partial_{\mu}A^{\gamma} B^{\zeta} + \partial_{\mu}B^{\zeta} A^{\gamma} \)
\end{equation}

** three term product rule

\begin{equation}
\partial_{\mu} \( A^{\gamma}B^{\zeta}G^{\alpha}\) \rightarrow
\( \partial_{\mu}A^{\gamma} B^{\zeta} G^{\alpha} + \partial_{\mu}B^{\zeta} G^{\alpha} A^{\gamma} + \partial_{\mu}G^{\alpha} B^{\zeta} A^{\gamma} \)
\end{equation}

*** Distribute through multiple terms
\begin{equation}
\partial_{\mu} \( A^{\gamma}B^{\zeta}\) \partial_{\alpha}\(A_{\gamma}\) + \partial_{\alpha}\(G_{\mu \zeta}\) \rightarrow
\( \partial_{\mu}A^{\gamma} B^{\zeta} + \partial_{\mu}B^{\zeta} A^{\gamma} \)\( \partial_{\alpha}A_{\gamma} \)+\( \partial_{\alpha}G_{\mu \zeta} \)
\end{equation}

** multiple partials product rule
\begin{equation}
\partial_{\alpha}\partial_{\gamma} \( A^{\alpha} B^{\gamma}\) \rightarrow
 \( \partial_{\alpha}\partial_{\gamma}A^{\alpha} B^{\gamma} + \partial_{\gamma}B^{\gamma} \partial_{\alpha}A^{\alpha} + \partial_{\alpha}\partial_{\gamma}B^{\gamma} A^{\alpha} + \partial_{\gamma}A^{\alpha} \partial_{\alpha}B^{\gamma} \)
\end{equation}

** product rule with squares\\

\begin{equation}
\square \(A^{\alpha}B_{\alpha}\) 
\rightarrow
\( \square A^{\alpha} B_{\alpha} + \partial^{\tau} B_{\alpha} \partial_{\tau} A^{\alpha} + \square B_{\alpha} A^{\alpha} + \partial^{\tau} A^{\alpha} \partial_{\tau} B_{\alpha} \)
\end{equation}


******** also check with equals signssssss\\
It appears that numbers are working but symbolic coefficients are not\\
** non-tensor terms\\
** only partial terms\\

ALSO test product rule with multiple tensors!!!!\\

QUESTION: why num co, symco, AND tensorCos?? What happens with multiple coefficients???\\

%%%
 \subsection{FOIL out terms without distributing derivatives}
Basic case: 
\begin{equation}
\(A^{}\) \(Y \partial_{\nu}h^{\mu \nu} + X \partial^{\mu} h^{\nu }_{\nu}\)  \rightarrow \(Y A^{} \partial_{\nu}h^{\mu \nu} +X A^{} \partial^{\mu}h_{\nu}^{\nu} \)
\end{equation}

Alternate cases:\\
*** constants that can move through the derivatives\\
\begin{equation}
\partial_{\mu} A \(Y \partial_{\nu}h^{\mu \nu} + X \partial^{\mu} h^{\nu }_{\nu}\)   \rightarrow \partial_{\mu}\(YA \partial_{\nu}h^{\mu \nu} +XA \partial^{\mu}h_{\nu}^{\nu} \)
\end{equation}


\begin{equation}
A \partial_{\mu} \(Y \partial_{\nu}h^{\mu \nu} + X \partial^{\mu} h^{\nu }_{\nu}\) 	
\rightarrow
\partial_{\mu}\(YA \partial_{\nu}h^{\mu \nu} +XA \partial^{\mu}h_{\nu}^{\nu} \)
\end{equation}


\begin{equation}
4 \partial_{\mu}  \(Y \partial_{\nu}h^{\mu \nu} + X \partial^{\mu} h^{\nu }_{\nu}\)   \rightarrow \partial_{\mu}\(4 Y \partial_{\nu}h^{\mu \nu} +4 X \partial^{\mu}h_{\nu}^{\nu} \)
\end{equation}

*** multiple terms
\begin{equation}
\(A + 5B + CD \) \partial_{\mu} \(Y \partial_{\nu}h^{\mu \nu} + X \partial^{\mu} h^{\nu }_{\nu}\) \rightarrow
\partial_{\mu}\(YA \partial_{\nu}h^{\mu \nu} +5 YB \partial_{\nu}h^{\mu \nu} +YCD \partial_{\nu}h^{\mu \nu} +XA \partial^{\mu}h_{\nu}^{\nu} +5 XB \partial^{\mu}h_{\nu}^{\nu} +XCD \partial^{\mu}h_{\nu}^{\nu} \)
\end{equation}

*** etas and deltas
\begin{equation}
\delta^{\gamma}_{\zeta} \eta^{\nu \alpha} \partial_{\mu} \(Y \partial_{\nu}h^{\mu \zeta} + X \partial^{\mu} h^{\zeta }_{\nu}\) \rightarrow
\partial_{\mu}\(Y \delta_{\zeta}^{\gamma} \eta^{\nu \alpha} \partial_{\nu}h^{\mu \zeta} +X \delta_{\zeta}^{\gamma} \eta^{\nu \alpha} \partial^{\mu}h_{\nu}^{\zeta} \)
\end{equation}

*** sum of mixed constants
\begin{multline}
\(A \eta^{\gamma \epsilon} + A5B + 56 \delta^{\phi}_{\xi} \) \partial_{\mu} \(Y \partial_{\nu}h^{\mu \nu} + X \partial^{\mu} h^{\nu }_{\nu}\)\\
 \rightarrow
\partial_{\mu} \(YA \eta^{\gamma \epsilon} \partial_{\nu} h^{\mu \nu} +5 YAB \partial_{\nu} h^{\mu \nu} +56 Y \delta_{\xi}^{\phi} \partial_{\nu} h^{\mu \nu} +XA \eta^{\gamma \epsilon} \partial^{\mu} h_{\nu}^{\nu} +5 XAB \partial^{\mu} h_{\nu}^{\nu} \\
+56 X \delta_{\xi}^{\phi} \partial^{\mu} h_{\nu}^{\nu} \)
\end{multline}

TODO: decide on how partials behave then implement test cases for them\\
Question: do all nodes have summation objects or do some have multgroup objects?? – they appear to all be summation objects\\
*** derivative is attached to a multgroup and doesn’t need to be distributed\\
\begin{equation}
 \partial_{\gamma}A^{\gamma}  \(Y \partial_{\mu}\partial_{\nu}h^{\mu \nu} + X \partial_{\mu}\partial^{\mu} h^{\nu }_{\nu}\)  \rightarrow \(Y \partial_{\gamma}A^{\gamma} \partial_{\mu}\partial_{\nu}h^{\mu \nu} +X \partial_{\gamma}A^{\gamma} \partial_{\mu}\partial^{\mu}h_{\nu}^{\nu} \)
\end{equation}

** with added term at the end
\begin{equation}
\partial_{\mu}   A^{\epsilon} \(Y \partial_{\nu}h^{\mu \nu} + X \partial^{\mu} h^{\nu }_{\nu}\) + B^{\epsilon} \rightarrow  \(Y \partial_{\mu}A^{\epsilon} \partial_{\nu}h^{\mu \nu} +X \partial_{\mu}A^{\epsilon} \partial^{\mu}h_{\nu}^{\nu} + B^{\epsilon} \)
\end{equation}


***with subtracted term at the end
\begin{equation}
\partial_{\mu}   A^{\epsilon} \(Y \partial_{\nu}h^{\mu \nu} + X \partial^{\mu} h^{\nu }_{\nu}\) - B^{\epsilon} \rightarrow  \(Y \partial_{\mu}A^{\epsilon} \partial_{\nu}h^{\mu \nu} +X \partial_{\mu}A^{\epsilon} \partial^{\mu}h_{\nu}^{\nu} - B^{\epsilon} \)
\end{equation}


*** test FOIL ability
** 2 terms
\begin{equation}
\(A^{\epsilon} + B^{\gamma} \) \( \partial_{\mu}X^{\mu} + M^{\nu \zeta}_{\gamma} \) \rightarrow
\( A^{\epsilon} \partial_{\mu}X^{\mu} + A^{\epsilon} M_{\gamma}^{\nu \zeta} + B^{\gamma} \partial_{\mu}X^{\mu} + B^{\gamma} M_{\gamma}^{\nu \zeta} \)
\end{equation}


** 3 terms
\begin{equation}
H \(A^{\epsilon} + B^{\gamma} \) \( \partial_{\mu}X^{\mu} + M^{\nu \zeta}_{\gamma} \) \( X + Y\) \rightarrow
\(HX A^{\epsilon} \partial_{\mu}X^{\mu} +HY A^{\epsilon} \partial_{\mu}X^{\mu} +HX A^{\epsilon} M_{\gamma}^{\nu \zeta} +HY A^{\epsilon} M_{\gamma}^{\nu \zeta} +HX B^{\gamma} \partial_{\mu}X^{\mu} +HY B^{\gamma} \partial_{\mu}X^{\mu} +HX B^{\gamma} M_{\gamma}^{\nu \zeta} \\
+HY B^{\gamma} M_{\gamma}^{\nu \zeta} \)
\end{equation}

TODO figure out the part with switching and write test cases for it\\
ALSO etas and deltas\\
** multgroup also has constant factor\\
\begin{equation}
B \partial_{\gamma}A^{\gamma}  \(Y \partial_{\mu}\partial_{\nu}h^{\mu \nu} + X \partial_{\mu}\partial^{\mu} h^{\nu }_{\nu}\)  \rightarrow  \(BY \partial_{\gamma}A^{\gamma} \partial_{\mu}\partial_{\nu}h^{\mu \nu} +BX \partial_{\gamma}A^{\gamma} \partial_{\mu}\partial^{\mu}h_{\nu}^{\nu} \)
\end{equation}

**** Shouldn’t be distributed
\begin{equation}
A^{\zeta} \partial_{\mu}  \(Y \partial_{\nu}h^{\mu \nu} + X \partial^{\mu} h^{\nu }_{\nu}\) \rightarrow  A^{\zeta} \partial_{\mu}\(Y \partial_{\nu}h^{\mu \nu} +X \partial^{\mu}h_{\nu}^{\nu} \)
\end{equation}

\begin{equation}
 \partial_{\mu}  \(Y \partial_{\nu}h^{\mu \nu} + X \partial^{\mu} h^{\nu }_{\nu}\) +  T^{\gamma} \(A_{\gamma} + B_{\gamma}\) \rightarrow \partial_{\mu}\(Y \partial_{\nu}h^{\mu \nu} +X \partial^{\mu}h_{\nu}^{\nu} \)+\( T^{\gamma} A_{\gamma} + T^{\gamma} B_{\gamma} \)
\end{equation}

*** with added term at the end
\begin{equation}
A^{\epsilon} \partial_{\mu}  \(Y \partial_{\nu}h^{\mu \nu} + X \partial^{\mu} h^{\nu }_{\nu}\) + B^{\epsilon}  \rightarrow A^{\epsilon} \partial_{\mu}\(Y \partial_{\nu}h^{\mu \nu} +X \partial^{\mu}h_{\nu}^{\nu} \)+ B^{\epsilon}
\end{equation}

*** with subtracted term at the end
\begin{equation}
A^{\epsilon} \partial_{\mu}  \(Y \partial_{\nu}h^{\mu \nu} + X \partial^{\mu} h^{\nu }_{\nu}\) - B^{\epsilon}  \rightarrow A^{\epsilon} \partial_{\mu}\(Y \partial_{\nu}h^{\mu \nu} +X \partial^{\mu}h_{\nu}^{\nu} \)- B^{\epsilon}
\end{equation}

{\color{blue}
Is this sensical or non-sensical???
\begin{equation}
\( \partial_{alpha} \) \(A^{\gamma}B_{\gamma}\) -> LEADS TO 
\partial_{alpha}\( A^{\gamma} B_{\gamma} \)
\end{equation}

What about…

\begin{equation}
\( \partial_{\alpha} \) \partial^{\alpha}\(A^{\gamma}B_{\gamma}\) -> LEADS TO
\partial_{\alpha}\partial^{\alpha}\( A^{\gamma} B_{\gamma} \)
\end{equation}
}

Also need to test adding summations, with and without partials, and subtracting summations, with and without partials\\



%%%%%%%%%%%%%%%%%%%%%%%%%%%%%%%%%%%%%%%%%%%%

\section{Contract}

%%%

 \subsection{contract etas and deltas}
\begin{equation}
\delta^{\gamma}_{\beta} \eta^{\nu \alpha} \partial_{\mu} h^{\mu}_{\alpha}\partial_{\nu} h^{\beta}_{\gamma}  \rightarrow  \partial_{\mu}h^{\mu \nu} \partial_{\nu}h_{\gamma}^{\gamma}
\end{equation}

%%%

 \subsection{contract only deltas}

Basic case:
\begin{equation}
\delta^{\nu }_{\alpha} \partial_{\mu} h^{\mu}_{\alpha}\partial_{\nu} h^{\gamma}_{\gamma}  \rightarrow \partial_{\mu}h_{\alpha}^{\mu} \partial_{\alpha}h_{\gamma}^{\gamma}
\end{equation}

%%%

 \subsection{contract only etas}

Basic case:
\begin{equation}
\eta^{\nu \alpha} \partial_{\mu} h^{\mu}_{\alpha}\partial_{\nu} h^{\gamma}_{\gamma} 	 \rightarrow  \partial_{\mu}h^{\mu \nu} \partial_{\nu}h_{\gamma}^{\gamma}
\end{equation}

Alternative cases:











%%%%%%%%%%%%%%%%%%%%%%%%%%%%%%%%%%%%%%%%%%%%

\section{Factor}

%%%

 \subsection{factor out GCF}

Basic case: 
\begin{equation}
 \(3Y \partial_{\nu}h^{\mu \nu} + Y \partial^{\mu} h^{\nu }_{\nu}\) \rightarrow
 Y \(3 \partial_{\nu}h^{\mu \nu} + \partial^{\mu}h_{\nu}^{\nu} \)
\end{equation}

Alternative cases:
\begin{equation}
\(X \partial_{\nu}h^{\mu \nu} + Y \partial_{\nu} h^{\mu \nu}\) \rightarrow
\partial_{\nu}h^{\mu \nu} \(X +Y \)
\end{equation}

*** sum
\begin{equation}
\(X A^{\alpha}_{\alpha} + Y A^{\alpha}_{\alpha}\) \rightarrow
A_{\alpha}^{\alpha} \(X +Y \)
\end{equation}

***  no outer brackets…
\begin{equation}
X A^{\alpha}_{\alpha} + Y A^{\alpha}_{\alpha}  \rightarrow
A_{\alpha}^{\alpha} \(X +Y \)
\end{equation}

\begin{equation}
\(X \partial_{\gamma}h^{\mu \gamma} + Y \partial_{\nu} h^{\mu \nu}\) \rightarrow
\partial_{\gamma}h^{\mu \gamma} \(X +Y \)
\end{equation}


\begin{equation}
\(X \partial_{\gamma}h^{\mu \gamma} A^{\alpha}_{\alpha} + Y \partial_{\nu} h^{\mu \nu}\) \rightarrow
 \partial_{\gamma}h^{\mu \gamma} \(X A_{\alpha}^{\alpha} +Y \)
\end{equation}

\begin{equation}
\(X \partial_{\gamma}h^{\mu \gamma} + Y h^{\gamma \nu}\) \rightarrow
\(X \partial_{\gamma}h^{\mu \gamma} +Y h^{\gamma \nu} \)
\end{equation}

\begin{equation}
\(\frac{1}{2} \partial_{\gamma}h^{\mu \gamma} + \frac{1}{4}  h^{\gamma \nu}\) \rightarrow
\frac{1}{4} \(2 \partial_{\gamma}h^{\mu \gamma} + h^{\gamma \nu} \)
\end{equation}

*** factor of 1 (implied)
\begin{equation}
\(X \partial_{\nu}h^{\mu \nu} + \partial_{\nu} h^{\mu \nu}\) \rightarrow
\partial_{\nu}h^{\mu \nu} \(X +1 \)
\end{equation}

%%%

\subsection{factor out user specified term}

*** smaller numerical
\begin{equation}
\(\frac{1}{2} \partial_{\gamma}h^{\mu \gamma} + \frac{1}{4}  h^{\gamma \nu}\) , \frac{1}{8} \rightarrow
\frac{1}{8} \(4 \partial_{\gamma}h^{\mu \gamma} +2 h^{\gamma \nu} \)
\end{equation}

*** bigger numerical

\begin{equation}
\(\frac{1}{2} \partial_{\gamma}h^{\mu \gamma} + \frac{1}{4}  h^{\gamma \nu}\) , 8 \rightarrow
8 \(\frac{1}{16} \partial_{\gamma}h^{\mu \gamma} +\frac{1}{32} h^{\gamma \nu} \)
\end{equation}


*** smaller tensor\\

\begin{equation}
\(X \partial_{\gamma}h^{\mu \gamma} A^{\gamma} + Y \partial_{\nu} h^{\mu \nu}A^{\gamma}\) , A^{\gamma} \rightarrow
A^{\gamma} \(X \partial_{\gamma}h^{\mu \gamma} +Y \partial_{\nu}h^{\mu \nu} \)
\end{equation}


*** not included tensor\\

\begin{equation}
X \partial_{\mu} \partial_{\nu}h^{\mu \nu} +Y \partial_{\nu}h^{\mu \nu}  , A^{alpha} \rightarrow
\(X \partial_{\mu}\partial_{\nu}h^{\mu \nu} +Y \partial_{\nu}h^{\mu \nu} \)
\end{equation}


*** different index\\

(***decide if this should be the expected behaviour (tensor logic question)) 
\begin{equation}
\(X \partial_{\gamma}h^{\mu \gamma} A^{\gamma} +Y \partial_{\nu}h^{\mu \nu} A^{\gamma} \), A^{\alpha} \rightarrow 
\(X \partial_{\gamma}h^{\mu \gamma} A^{\gamma} +Y \partial_{\nu}h^{\mu \nu} A^{\gamma} \)
\end{equation}

*** make sure it’s recognizing not to factor out tensors under partials\\
\begin{equation}
\(X \partial_{\mu} \partial_{\nu}h^{\mu \nu} +Y \partial_{\nu}h^{\mu \nu} A^{\gamma} \), \partial_{\nu}h^{\mu \nu} \rightarrow
\partial_{\nu}h^{\mu \nu} \(Y A^{\gamma} \)+\(X \partial_{\mu}\partial_{\nu}h^{\mu \nu} \)
\end{equation}

*** what about if there’s a different index with the same sum pattern, and what about coefficient of 1\\
\begin{equation}
\(X A^{\alpha}_{\alpha} + A^{\alpha}_{\alpha}\) , A^{\beta}_{\beta} \rightarrow
A_{\beta}^{\beta} \(X +1 \)
\end{equation}


— variation, no brackets…\\
\begin{equation}
X A^{\alpha}_{\alpha} + A^{\alpha}_{\alpha} , A^{\beta}_{\beta} \rightarrow
A_{\beta}^{\beta} \(X +1 \)
\end{equation}


%%%%%%%%%%%%%%%%%%%%%%%%%%%%%%%%%%%%%%%%%%%%

\section{Replace}

%%%

\subsection{replace indices}

Basic case:\\

*** single replacement
\begin{equation}
4X \partial_{\mu} \partial_{\nu}h^{\mu \nu} +Y \partial_{\nu}h^{\mu \nu} A^{\gamma} , \mu \rightarrow \alpha \rightarrow 
\(4 X \partial_{\alpha}\partial_{\nu}h^{\alpha \nu} +Y \partial_{\nu}h^{\alpha \nu} A^{\gamma} \)
\end{equation}

*** list replacement
\begin{equation}
4X \partial_{\mu} \partial_{\nu}h^{\mu \nu} +Y \partial_{\nu}h^{\mu \nu} A^{\gamma} , \mu, \nu, \gamma \rightarrow \alpha, \beta, \xi \rightarrow 
 \(4 X \partial_{\alpha}\partial_{\beta}h^{\alpha \beta} +Y \partial_{\beta}h^{\alpha \beta} A^{\xi} \)
\end{equation}

*** replace index that doesn’t exist
\begin{equation}
4X \partial_{\mu} \partial_{\nu}h^{\mu \nu} +Y \partial_{\nu}h^{\mu \nu} A^{\gamma} , \mu, \nu, \xi \rightarrow \alpha, \beta, \chi \rightarrow 
\(4 X \partial_{\alpha}\partial_{\beta}h^{\alpha \beta} +Y \partial_{\beta}h^{\alpha \beta} A^{\gamma} \)
\end{equation}

*** what if it’s not there but it’s one of the replacement indices
\begin{equation}
4X \partial_{\mu} \partial_{\nu}h^{\mu \nu} +Y \partial_{\nu}h^{\mu \nu} A^{\gamma} , \mu, \nu, \alpha \rightarrow \alpha, \beta, \xi \rightarrow 
\(4 X \partial_{\alpha}\partial_{\beta}h^{\alpha \beta} +Y \partial_{\beta}h^{\alpha \beta} A^{\gamma} \)
\end{equation}

*** long number of indices (mismatched list lengths) 


%%%

\subsection{replace terms}

Basic case:

*** direct replacement
\begin{equation}
4X \partial_{\mu} \partial_{\nu}h^{\mu \nu} +Y \partial_{\nu}h^{\mu \nu} A^{\gamma} , A^{\gamma} \rightarrow B^{\gamma} \rightarrow  
\(4 X \partial_{\mu}\partial_{\nu}h^{\mu \nu} +Y \partial_{\nu}h^{\mu \nu} B^{\gamma} \)
\end{equation}

{\color{red}
*** more complicated replacement term
\begin{equation}
4X \partial_{\mu} \partial_{\nu}h^{\mu \nu} +Y \partial_{\nu}h^{\mu \nu} A^{\gamma} ,  \partial_{\nu}h^{\mu \nu} \rightarrow \partial_{\gamma} V^{\mu \gamma}  ISSUE:" Equation: indices to replace do not match in length!
Please email jbrucero@uwo.ca for if you think this is a bug"
\end{equation}
}

{\color{red}
\begin{equation}
4X \partial_{\mu} \partial_{\nu}h^{\mu \nu} +Y \partial_{\nu}h^{\mu \nu} A^{\gamma} ,  \partial_{\nu}h^{\mu \nu} \rightarrow  V^{\mu} \rightarrow ISSUE:" Equation: indices to replace do not match in length!
Please email jbrucero@uwo.ca for if you think this is a bug"
\end{equation}
}


— ***probably one of the most complicated algorithms, need to look over in detail then finish cases***\\




%%%%%%%%%%%%%%%%%%%%%%%%%%%%%%%%%%%%%%%%%%%%

\section{Sort}

%%%

\subsection{ combine like terms differing only by a numerical factor}

Basic case:
\begin{equation}
3A^{\gamma} + \frac{5}{7} A^{\gamma} \rightarrow 
\(\frac{26}{7} A^{\gamma} \)
\end{equation}

*** fraction simplification
\begin{equation}
3A^{\gamma} + \frac{4}{2} A^{\gamma}  \rightarrow 
\(5 A^{\gamma} \)
\end{equation}

*** coefficient of 1 (implied)
\begin{equation}
A^{\gamma} + A^{\gamma} \rightarrow 
\(2 A^{\gamma} \)
\end{equation}

*** more complicated term
\begin{equation}
3\partial_{\gamma}A^{\gamma} + \frac{5}{7} \partial_{\beta}A^{\beta} \rightarrow 
\(\frac{26}{7} \partial_{\gamma}A^{\gamma} \)
\end{equation}

\begin{equation}
3\partial_{\gamma}\partial^{\mu}A^{\gamma}_{\nu \mu} + 7 \partial_{\beta}\partial^{\zeta}A^{\beta}_{\nu \zeta} \rightarrow 
 \(10 \partial_{\gamma}\partial^{\mu}A_{\nu \mu}^{\gamma} \)
\end{equation}

*** not all terms combine 
\begin{equation}
3\partial_{\gamma}A^{\gamma} + 4 A^{\chi}+ \frac{5}{7} \partial_{\beta}A^{\beta} \rightarrow 
 \(\frac{26}{7} \partial_{\gamma}A^{\gamma} +4 A^{\chi} \)
\end{equation}

*** things that shouldn’t combine\\

*** different tensors
\begin{equation}
3A^{\gamma} + \frac{5}{7} B^{\gamma} \rightarrow  
\(3 A^{\gamma} +\frac{5}{7} B^{\gamma} \)
\end{equation}

*** not differing only by numerical factor
\begin{equation}
3XA^{\gamma} + 7 A^{\gamma}  \rightarrow 
\(3 X A^{\gamma} +7 A^{\gamma} \)
\end{equation}

***different free indices
\begin{equation}
A^{\beta} + A^{\alpha} \rightarrow 
\( A^{\beta} + A^{\alpha} \)
\end{equation}

***different number of partials
\begin{equation}
3\partial_{\gamma}A^{\gamma}_{\nu} + 7 \partial_{\beta}\partial^{\zeta}A^{\beta}_{\nu \zeta} \rightarrow 
\(3 \partial_{\gamma}A_{\nu}^{\gamma} +7 \partial_{\beta}\partial^{\zeta}A_{\nu \zeta}^{\beta} \)
\end{equation}

*** different position of free index
\begin{equation}
3\partial_{\gamma}\partial^{\mu}A^{\gamma}_{\nu \mu} + 7 \partial_{\beta}\partial^{\zeta}A^{\beta}_{\zeta \nu} \rightarrow 
\(3 \partial_{\gamma}\partial^{\mu}A_{\nu \mu}^{\gamma} +7 \partial_{\beta}\partial^{\zeta}A_{\zeta \nu}^{\beta} \)
\end{equation}

***(Unless A is a symmetric tensor, in which case):
\begin{equation}
3\partial_{\gamma}\partial^{\mu}A^{\gamma}_{\nu \mu} + 7 \partial_{\beta}\partial^{\zeta}A^{\beta}_{\zeta \nu} \rightarrow 
\(10 \partial_{\gamma}\partial^{\mu}A_{\nu \mu}^{\gamma} \)
\end{equation}


%%%

\subsection{ combine like terms differing by any (numerical or symbolic) coefficient}


*** basic case
\begin{equation}
3XA^{\gamma} + 7 A^{\gamma} \rightarrow 
\(\(3X+7\) A^{\gamma} \)
\end{equation}

*** more complicated term

\begin{equation}
3B \partial_{\gamma}A^{\gamma} + \frac{5}{7} V \partial_{\beta}A^{\beta} \rightarrow 
\(\(3B+\frac{5}{7}V\) \partial_{\gamma}A^{\gamma} \)
\end{equation}

\begin{equation}
3B \partial_{\gamma}A^{\gamma} + \frac{5}{7} B \partial_{\beta}A^{\beta} \rightarrow 
\(\frac{26}{7}B \partial_{\gamma}A^{\gamma} \)
\end{equation}

\begin{equation}
3M\partial_{\gamma}\partial^{\mu}A^{\gamma}_{\nu \mu} + 7X \partial_{\beta}\partial^{\zeta}A^{\beta}_{\nu \zeta} \rightarrow 
\(\(3M+7X\) \partial_{\gamma}\partial^{\mu}A_{\nu \mu}^{\gamma} \)
\end{equation}

*** not all terms combine 

\begin{equation}
3V\partial_{\gamma}A^{\gamma} + 4 ZA^{\chi}+ \frac{5}{7} N \partial_{\beta}A^{\beta} \rightarrow 
\(\(3V+\frac{5}{7}N\) \partial_{\gamma}A^{\gamma} +4Z A^{\chi} \)
\end{equation}

***shouldn’t combine
\begin{equation}
3B \partial_{\gamma}A^{\gamma} + \frac{5}{7} V^{\alpha}_{\alpha} \partial_{\beta}A^{\beta} \rightarrow  (I think this is what I want but should double check/ see if an improvement combine these too \rightarrow would this be helpful?)
\(3B \partial_{\gamma}A^{\gamma} +\frac{5}{7} V_{\alpha}^{\alpha} \partial_{\beta}A^{\beta} \)
\end{equation}

*** different tensors
\begin{equation}
3CA^{\gamma} + \frac{5}{7} DB^{\gamma} \rightarrow  
\(3C A^{\gamma} +\frac{5}{7}D B^{\gamma} \)
\end{equation}

***different free indices
\begin{equation}
XA^{\beta} + A^{\alpha} \rightarrow 
\(X A^{\beta} + A^{\alpha} \)
\end{equation}

***different number of partials
\begin{equation}
3\partial_{\gamma}A^{\gamma}_{\nu} + 7M \partial_{\beta}\partial^{\zeta}A^{\beta}_{\nu \zeta} \rightarrow 
\(3 \partial_{\gamma}A_{\nu}^{\gamma} +7M \partial_{\beta}\partial^{\zeta}A_{\nu \zeta}^{\beta} \)
\end{equation}

*** different position of free index
\begin{equation}
3V\partial_{\gamma}\partial^{\mu}A^{\gamma}_{\nu \mu} + 7L \partial_{\beta}\partial^{\zeta}A^{\beta}_{\zeta \nu} \rightarrow 
\(3V \partial_{\gamma}\partial^{\mu}A_{\nu \mu}^{\gamma} +7L \partial_{\beta}\partial^{\zeta}A_{\zeta \nu}^{\beta} \)
\end{equation}

***(Unless A is a symmetric tensor, in which case):
\begin{equation}
3V\partial_{\gamma}\partial^{\mu}A^{\gamma}_{\nu \mu} + 7L \partial_{\beta}\partial^{\zeta}A^{\beta}_{\zeta \nu} \rightarrow 
\(\(3V+7L\) \partial_{\gamma}\partial^{\mu}A_{\nu \mu}^{\gamma} \)
\end{equation}

%%%

 \subsection{sort the tensors in each term by number of derivatives (least to greatest)}

\begin{equation}
\partial_{\gamma} \square G^{\nu \gamma} \partial_{\beta}\partial^{\xi} M^{\beta}_{\xi}   \square \square \partial_{\chi} X^{\kappa \zeta}_{\nu}  \partial_{\mu}T^{\mu} \rightarrow 
\partial_{\mu}T^{\mu} \partial_{\beta}\partial^{\xi}M_{\xi}^{\beta} \partial_{\gamma}\square G^{\nu \gamma} \partial_{\chi}\square \square X_{\nu}^{\kappa \zeta}
\end{equation}


***What about with coefficients
\begin{equation}
4A\partial_{\gamma} \square G^{\nu \gamma} \partial_{\beta}\partial^{\xi} M^{\beta}_{\xi}   \square \square \partial_{\chi} X^{\kappa \zeta}_{\nu}  \partial_{\mu}T^{\mu} \rightarrow WORKING
4 A \partial_{\mu}T^{\mu} \partial_{\beta}\partial^{\xi}M_{\xi}^{\beta} \partial_{\gamma}\square G^{\nu \gamma} \partial_{\chi}\square \square X_{\nu}^{\kappa \zeta}
\end{equation}

***What about multiple terms and coefficients
\begin{equation}
4A\partial_{\gamma} \square G^{\nu \gamma} \partial_{\beta}\partial^{\xi} M^{\beta}_{\xi} +
  \frac{7}{8} C \square \square \partial_{\chi} X^{\kappa \zeta}_{\nu}  \partial_{\mu}T^{\mu} \rightarrow 
\(4 A \partial_{\beta}\partial^{\xi}M_{\xi}^{\beta} \partial_{\gamma}\square G^{\nu \gamma} +\frac{7}{8} C \partial_{\mu}T^{\mu} \partial_{\chi}\square \square X_{\nu}^{\kappa \zeta} \)
\end{equation}

***And with brackets 
\begin{equation}
\( 4A\partial_{\gamma} \square G^{\nu \gamma} \partial_{\beta}\partial^{\xi} M^{\beta}_{\xi}\)\(
  \frac{7}{8} C \square \square \partial_{\chi} X^{\kappa \zeta}_{\nu}  \partial_{\mu}T^{\mu} \) \rightarrow 
\(4 A \partial_{\beta}\partial^{\xi}M_{\xi}^{\beta} \partial_{\gamma}\square G^{\nu \gamma} \)\(\frac{7}{8} C \partial_{\mu}T^{\mu} \partial_{\chi}\square \square X_{\nu}^{\kappa \zeta} \)
 \end{equation}

%%%

 \subsection{sort terms by number of derivatives (least to greatest)}

\begin{equation}
\partial_{\gamma} \partial_{\kappa}A^{\gamma} + \partial^{\chi}B_{\chi} + C^{} \rightarrow 
\( C^{} + \partial^{\chi}B_{\chi} + \partial_{\gamma}\partial_{\kappa}A^{\gamma} \)
\end{equation}

***with brackets
\begin{equation}
\(\partial_{\gamma} \partial_{\kappa}A^{\gamma} + \partial^{\chi}B_{\chi} + C^{}\)  \rightarrow 
\( C^{} + \partial^{\chi}B_{\chi} + \partial_{\gamma}\partial_{\kappa}A^{\gamma} \)
\end{equation}

***with coefficients
\begin{equation}
\(  9\partial^{\chi}B_{\chi} +\frac{1}{2}T \partial_{\gamma}\partial_{\kappa}A^{\gamma} +M C^{} \) \rightarrow 
\(M C^{} +9 \partial^{\chi}B_{\chi} +\frac{1}{2} T \partial_{\gamma}\partial_{\kappa}A^{\gamma} \)
\end{equation}

*** with multiplied terms

\begin{equation}
\(\partial_{\gamma} \partial_{\kappa}A^{\gamma}  + \partial_{\omega}G^{} \)\( \partial^{\chi}B_{\chi} + C^{} \) \rightarrow 
\( \partial_{\omega}G^{} + \partial_{\gamma}\partial_{\kappa}A^{\gamma} \)\( C^{} + \partial^{\chi}B_{\chi} \)
\end{equation}

** with multiple tensors per term (I think it goes by either least or greatest…)\\

\begin{equation}
\(\partial_{\gamma} \partial_{\kappa}\square A^{\gamma}\partial_{\omega}G^{} + \square \partial^{\chi}B_{\chi} + \square X_{\xi}+ \square V_{}C^{} \) \rightarrow 
\( \square X_{\xi} + \square V^{} C^{} + \partial^{\chi}\square B_{\chi} + \partial_{\gamma}\partial_{\kappa}\square A^{\gamma} \partial_{\omega}G^{} \)
\end{equation}

(must sort by total)\\
\begin{equation}
 \partial_{\xi}H^{}\partial_{\omega}G^{}\partial^{\xi}M^{} + \square A^{} \rightarrow 
\( \square A^{} + \partial_{\xi}H^{} \partial_{\omega}G^{} \partial^{\xi}M^{} \)
\end{equation}

*** interesting thing with brackets - should look into at some point\\
-> At the very least it should be a warning and not make it look like it worked...\\


*** same idea but with summation -> INTERESTING BRACKET STUFF - is it important??\\

{\color{red}
\begin{equation}
(X A^{\alpha}_{\alpha} + Y A^{\alpha}_{\alpha}) \rightarrow NOT WORKING
(X + Y ) A_{\alpha}^{\alpha} A_{\alpha}^{\alpha}
\end{equation}
}

— variations on above:
{\color{red}
\begin{equation}
(X \partial_{\mu}A^{\alpha}_{\alpha} + Y \partial_{\mu} A^{\alpha}_{\alpha}) \rightarrow same issue
(X + Y ) \partial_{\mu}A_{\alpha}^{\alpha} \partial_{\mu}A_{\alpha}^{\alpha}
\end{equation}
}

{\color{red}
\begin{equation} 
(X \partial_{\alpha}A^{\alpha} + Y \partial_{\alpha} A^{\alpha}) \rightarrow ISSUE
(X + Y ) \partial_{\alpha}A^{\alpha} \partial_{\alpha}A^{\alpha}
\end{equation}
}

\begin{equation}
(X \partial_{\alpha}h^{\alpha} + Y \partial_{\alpha} h^{\alpha}) \rightarrow (X + Y ) \partial_{\alpha}h^{\alpha} \partial_{\alpha}h^{\alpha}
\end{equation}


\begin{equation}
(X \partial_{\alpha}h^{\alpha \beta} + Y \partial_{\alpha} h^{\alpha \beta}) \rightarrow
(X + Y ) \partial_{\alpha}h^{\alpha \beta} \partial_{\alpha}h^{\alpha \beta}
\end{equation}








\end{document}